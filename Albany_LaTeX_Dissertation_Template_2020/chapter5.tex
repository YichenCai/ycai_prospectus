%%%%%%%%%%%%%%%%%%%%%%%%%%%%%%%%%%%%%%%%%%%%%%%%%%%%%%%%%%%%%%%%%%% 
%                                                                 %
%                            CHAPTER                              %
%                                                                 %
%%%%%%%%%%%%%%%%%%%%%%%%%%%%%%%%%%%%%%%%%%%%%%%%%%%%%%%%%%%%%%%%%%% 
 
\chapter{Case Study: Thunderstorm in Washington D.C. June 2016}
\label{chap:case}
\resetfootnote %this command starts footnote numbering with 1 again.
The above observational overview as well as the diagram in \ref{fig:data_timeline} guide the case study selection for the microphysical analysis of lightning-producing storms. So far, three thunderstorm cases have been simulated: A multi-cellular thunderstorm cluster in the Washington DC region in late June 2016, a quasi-linear convective system (QLCS) in north Alabama in early June 2012, and an idealized squall line featuring the thunderstorm that occurred during Mid-latitude Continental Convective Clouds Experiment (MC3E) field campaign in May 2011. This prospectus provides an initial analysis of the thunderstorm in DC region, and will serve as a guide for further analysis including, but not limited to, the remaining two cases.
%---------------------------------------------------------------------------------------%
\section{Overview}
\label{sec:case_ovv}
The timing of this thunderstorm coincided with the availability of the DCLMA network, providing an abundance of spatial and temporal information of lightning discharge. The storm initiated at around 14Z on June 21, 2016 in southwest Pennsylvania, intensified and expanded in the next 5 hours, and swept over the DC metropolitan area at 17-22Z, after which the system propagated eastward into the Atlantic Ocean and dissipated. In this section, synoptic discussion and mesoscale dissection of the event are included.

\subsection{Synoptics}
Surface analysis from the NOAA Weather Prediction Center (WPC) shows a cold front associated with a low pressure center trough extending from Great Lake region to north Missouri (\ref{fig:synoptics} (a)) at 2016-06-21 00Z was propagating eastward. The region east of the cold front experiences dynamic forcing by the cold front, conducive to convection initiation. By 12Z, the cold front slowed down its propagation as the warm coastal air pushes against the cooler continental air. The DC region is south of the quasi east-west-oriented front, undergoing weak low-level convergence (\ref{fig:synoptics} (d)). Starting at 12Z, the cold front became quasi-stationary and hovered  in southern Pennsylvania. Throughout the event, the DC region remained within the warm sector of the front, which favored low-level convergence and lifting. The upper air analysis (not shown here) indicates that the region of interest (\ref{fig:synoptics} (d), red box) experienced south-westerly low-level wind, advecting (weakly) in moisture. At 500mb, the DC region coincides with the center of the deep pressure trough, with weak warm advection below 500mb. 250mb analysis shows a jet stream spanning the U.S.-Canada boarder. The DC region was located slightly the right side of the jet stream exit, but not within the region containing the maximum upper-level divergence associated with the secondary vertical circulation of the jet stream. This synoptic condition is conducive to convection initiation, but not optimal.

\subsection{Mesoscale}
%---------------------------------------------------------------------------------------%
The 18Z sounding at Blacksburg, VA (\ref{fig:sounding}), just upstream of the DC region, exhibits a very moist boundary layer with a surface dew point of about 70$^\circ$F and large instability (CAPE=1878J/kg). The wind profile shows significant shear in wind speed, but insignificant directional shear (mostly westerlies). This wind shear structure is conducive for new cell generations by separation of updrafts and downdrafts and organization into multi-cellular linear convective system, but not favorable for the longevity of a single cell. 

\ref{fig:refs} shows the evolution of the thunderstorm, with the red box representing approximately the region simulated. At 14Z, new convective cells were generated at southwest Pennsylvania due to the weakly convergent synoptic background and, more importantly, daytime radiative heating. Also noticeable is that this location is where westerly wind meets the central-north part of Appalachian Mountains (\ref{fig:Domain_2s_20160621_dc}), where lifting due to topography can contribute to convective initiation. These cells moved eastward, with new cells continuing to develop to the west and along the quasi-stationary front. By 17Z (\ref{fig:refs}(c)), a east-west oriented quasi-linear cluster of convective cells was organized at the borders of Pennsylvania and Maryland, spanning $\sim$200km in length. Moving east-southward toward the region of interest, the western tail comprised of new and developing convective cells swept over the DC metropolitan area, generating significant lightning detected by DCLMA. This storm experienced a peak in lightning activity at 18-19Z with a ``bow-echo" radar signature, and reached the mature stage (with maximum reflectivity) at around 19:30Z, followed by dissipation as it proceeds into the ocean. The DC region experienced a secondary peak of lightning activity at 22-23Z, which came from another cluster of convective cells initiated at 19:30Z in northern tip of Virginia, but did not benefit from sufficient daytime heating to develop as much strength as the previous storm.

Overall, the storm has moderate intensity and is the leftover convection set up in the background of weak dynamic lifting by a synoptic front and topography. The convection was later triggered and intensified by surface adiabatic heating by solar radiation, and developed into quasi-linear multicellular morphology determined by a moderate-large shear in wind speed and insignificant directional shear. The storm was the result of combined factors, with difficulty drawing conclusion on a dominant one. Additionally, with a comparatively small spatial scale and mobile nature of the system, the predictability was significantly decreased.  In this prospectus, real data simulation is conducted to facilitate comparison with observation. However, the focus of the project is not infinite proximity to observation, but the sensitivity of storm electrification to the change in representation of ice-phase hydrometeors. 
%---------------------------------------------------------------------------------------%
\section{Model Setup}
 The case described above was simulated using WRF version 3.9.1.1. Simulations were run using two-way nested grids over the region demonstrated in \ref{fig:Domain_2s_20160621_dc} centered at $38.907^{\circ}N,77.037^{\circ}W$. Two horizontal domain resolution sets are used: (1) 16km for domain 1 and 4km for domain 2; (2) 4km for domain 1 and 1km for domain 2, with a detailed listed of simulations provided in \ref{table:config}. Common configurations among the simulations are: A 4:1 ratio of horizontal grid size; 40 vertical levels with interval increasing with height, extending to 100mb($\sim$16km); An outer domain extra spin-up time of 1 day prior to the onset of the inner domain (06-20 00Z or 06-19 00Z), and the same simulation ending time (06-22 10Z). The Dudhia Shortwave Scheme (\cite{dudhia1989numerical}), RRTM Longwave Scheme (\cite{mlawer1997radiative}), Revised MM5 Surface Layer Scheme (\cite{jimenez2012revised}), Unified Noah Land Surface Model (\cite{mukul2004implementation}), and YSU Planetary Boundary Layer scheme (\cite{hong2006new}) are used for both domains.
 
Attempts are made to simulate the overall event as realistic as possible, which is required to focus analysis on the electrification and microphysics and minimize issues raised beyond these factors. A variety of sensitivity simulations were run to test the sensitivity of storm location and intensity to: Initial/Boundary Conditions (IC/BC) based on different models (GFS, NAM), model initialization time (1 day difference in spin-up), horizontal/time resolution (4km/20s, 1km/5s for inner domain), and utilization of cumulus scheme (no scheme, Kain-Fritsch scheme). Detailed variations of the inner domain configuration are summarized in \ref{table:config}. Results showed that the location of storm was the most sensitive to horizontal/time resolution, and thus the simulations were separated into two groups: 1km and 4km for inner domain resolution. Within each group, there is a benchmark setup (\ref{table:config}, bold font) based on which five other members in the group are modified. Each member deviates from the control run of the group with only one factor. The setup of the two control runs were based on literature review (e.g., resolution higher than 3-4km can resolve convection without the cumulus scheme, IC/BC with higher resolution is beneficial to mesoscale simulation) and limitation on the M2M scheme with graupel (no hail, morr\_graup), which did not produce lightning.


From \ref{fig:radard02_21T12}, the 1km simulations overestimated or incorrectly located a cluster of echos that ``barged in" from the northwest corner of the domain at 06-21 02Z and traveled southward, which might result from a poor representation of a squall line that was triggered by the cold front earlier that day but was mostly passing through New York and northern Pennsylvania. However, the storm of interest (14-21Z in the DC region) was captured much better by the 1km runs, in terms of both location and intensity. The 4km runs, on the other hand, did not simulate the incorrect cluster as in the 1km runs, but was unable to simulate the storm of interest at all, or only much later. Hence, it appears that the 4km simulations represented the synoptic scale signature in closer proximity to observation, but the 1km simulation is required to decently capture the storm of interest. Thus, two alternatives will be considered: (1) A 3-domain simulation with horizontal resolutions of 16, 4, and 1km, and (2) the ``piggybacking" technique (\cite{grabowski2019separating}), where the 1km simulation will use output from the 4km run as IC/BC. In this way, the simulation will benefit from both a better representation of the synoptic and more accurately resolved mesoscale features. Note that the two initial simulations were opted fro computational efficiency due to the higher computational requirements when simulating electrification simultaneously. However, given the relatively poor and inconsistent representation of the storm, these less efficient methods will be tested with hopes of improved accuracy.