%%%%%%%%%%%%%%%%%%%%%%%%%%%%%%%%%%%%%%%%%%%%%%%%%%%%%%%%%%%%%%%%%%% 
%                                                                 %
%                            CHAPTER                              %
%                                                                 %
%%%%%%%%%%%%%%%%%%%%%%%%%%%%%%%%%%%%%%%%%%%%%%%%%%%%%%%%%%%%%%%%%%% 
 
\chapter{Motivation}
\resetfootnote %this command starts footnote numbering with 1 again.

Like magic of the nature and pulses of the atmosphere, the evanescent but vigorous strikes liven the earth as the nature liberates its power. Lightning itself is intriguing with its energy and beauty. 

Reasons for studying lightning are endless. According to \cite{holle2016number}, the estimated global lightning fatalities are close to 24,000/year, with an injury rate about 10 times as much. Nationwide, there are 20 year averaged loss  of \$1.02 billion/year and fatality rate of 51/year directly due to lightning, according to \href{https://www.weather.gov/safety/lightning-fatalities}{NWS}. This record is just below the impact of flash floods (30 Year Flood Loss Averages = \$7.96 Billion in damages/year, 82 fatalities/year, \cite{nws2018}). Moreover, lightning activity, though not always a maximum in regions with largest overall rainfall (\cite{holle2016lightning}), is found to be well correlated with extreme convective rainfall and flash floods (\cite{price2011using,lynn2010prediction,soula2001some,carte1977lightning,tapia1998estimation}). Indirectly, lightning in the US initiates approximately 22,600 wildfires/year, resulting in \$461 million/year of direct property damage, as reported by National Fire Protection Association (\cite{ahrens2013lightning}). With a warmer climate that almost guarantees to intensify lightning hazards, combined with larger metropolitan areas and increased population, humans are becoming even more vulnerable to lightning activity (\cite{yair2018lightning}).

Regardless of its negative impacts, lightning data from observation has been applied in many fields. Using correlation studies, lightning data has been used to help identify rapidly developing thunderstorms in South Africa (\cite{gijben2017using}), predict imminent intense convective rainfall and possible flash flood  (\cite{price2011using}), and has been used as an indicator of climate change (\cite{reeve1999lightning}), etc. Correlation-based lightning data applications, though, should be conducted with caution. As stated in \cite{solimine2021relationships} (in review), there are great complexities in lightning-precipitation relationships over the Congo, which varies with stages of storms and time of the day. This makes the physical understanding (e.g., through numerical modeling) of lightning generation important to the application of lightning data. 

Tremendous research has been done in recent years to unmask lightning discharge processes, but many questions of lightning (e.g., initiation processes) remain unanswered (\cite{mazur2016lightning}). Storm electrification is modeled through parameterizations of charging and discharge processes. As the initial and fundamental process, charging is usually coupled with hydrometeor interactions in microphysics parameterizations. These charging schemes are parameterized based on numerous laboratory studies (e.g., \cite{reynolds1957thunderstorm}; \cite{takahashi1978riming}; \cite{jayaratne1983laboratory};\cite{saunders1991effect};\cite{saunders1998laboratory}). Current charging schemes consider the temperature, cloud water content (CWC), ice particle rime accretion rate (RAR), and droplet size distribution. Many works have been completed to study the sensitivity of simulated storm electrification to concentration of cloud condensation nuclei and cloud ice nuclei (\cite{takahashi1984thunderstorm}; \cite{mansell2013aerosol}), lab-based functions used to determine charge transferred to graupel particles, inclusion of inductive charging process (\cite{mansell2005charge}), representation of Hallett-Mossop secondary ice multiplication (\cite{mansell2013aerosol}), inclusion of a 3-D branching algorithm in the discharge model (\cite{barthe2007simulation}), vertical velocity fields in different regions of the storms (\cite{wang2015impact}), etc. To my best knowledge, there still lacks an extensive study on how the representation of ice particle growth (e.g., ice crystal habit) can affect simulated storm electrification. Studies of the charging rate dependence to ice crystal size can be traced back to \cite{jayaratne1983laboratory}, where a drastic increase of charge transferred per event was found as ice crystal size increases. This dependence was further studied and formulated in \cite{keith1989charge,
keith1990further},concluding that the strong dependence of charging to crystal size is limited to smaller crystals, which implies the importance of pristine ice crystal representation for charge transfer. As stated in \cite{harrington2013methoda,harrington2013methodb}, a physical evolution of ice crystal shape (habit) is important for simulating more realistic ice crystal size and fall speed. Furthermore, \cite{baker1987influence} discovered the crucial role of ice particle vapor diffusional growth (VDG) rates in the determination of charge sign and magnitude upon particle collisions, where the VDG rate is a direct function of ice crystal habit (\cite{sulia2011ice}). Since the crystal habit is not a variable to be "determined" instantaneously, but evolved over time, it is potentially beneficial to apply a model that tracks the growth history (e.g., temperature, humidity, aspect ratio) of ice particles, which in turn feeds the charging scheme for a more realistic result upon particle interaction (similar perspectives expressed in \cite{keith1990further}). This brings out the objective of this research project: \textbf{Investigating the sensitivity of simulated storm electrification to ice microphysical representation in numerical models}.
