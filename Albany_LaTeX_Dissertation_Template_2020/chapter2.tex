%%%%%%%%%%%%%%%%%%%%%%%%%%%%%%%%%%%%%%%%%%%%%%%%%%%%%%%%%%%%%%%%%%% 
%                                                                 %
%                            CHAPTER                              %
%                                                                 %
%%%%%%%%%%%%%%%%%%%%%%%%%%%%%%%%%%%%%%%%%%%%%%%%%%%%%%%%%%%%%%%%%%% 
 
\chapter{Background}
\resetfootnote %this command starts footnote numbering with 1 again.

The vast majority of lightning occurs in cold/mixed phased clouds (\cite{wallace2006atmospheric}), where the interaction of ice phase hydrometeors is a fundamental ingredient for lightning production. It is generally accepted that the electrical structure of thunderstorms is developed by updrafts that separate hydrometeors (ice \& graupel esp.) with opposite charge signs into upper and lower levels (\cite{saunders2008charge}). The transfer of electric charge results from collisions of hydrometeors, and is highly sensitive to thermodynamic conditions within which they are grown. In most severe thunderstorms associated with intense updrafts and thus large vertical extent, a sufficient frozen hydrometeor growth zone (above \SI{0}{\celsius} level) and the charge zone (0$\sim$\SI{-20}{\celsius}) are almost guaranteed. The large vertical span also provides colorful growth environments which culture hydrometeors of various properties. This variety is the fundamental reason for differed charge density on particle surface among hydrometeor types (e.g., pristine ice crystals, snow aggregates, graupel), and subsequently a transfer of charge upon particle collisions in an attempt to neutralize the charge imbalance. 

%---------------------------------------------------------------------------------------%
\section{Charging Mechanisms}
%---------------------------------------------------------------------------------------%


%---------------------------------------------------------------------------------------%
\section{Role of Ice Crystal Growth}
%---------------------------------------------------------------------------------------%

Here is the content for section two.

%---------------------------------------------------------------------------------------%
\section{Role of Riming}
%---------------------------------------------------------------------------------------%