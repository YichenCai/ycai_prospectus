%%%%%%%%%%%%%%%%%%%%%%%%%%%%%%%%%%%%%%%%%%%%%%%%%%%%%%%%%%%%%%%%%%% 
%                                                                 %
%                            CHAPTER                              %
%                                                                 %
%%%%%%%%%%%%%%%%%%%%%%%%%%%%%%%%%%%%%%%%%%%%%%%%%%%%%%%%%%%%%%%%%%% 
 
\chapter{Background}
\resetfootnote %this command starts footnote numbering with 1 again.

The vast majority of lightning occurs in cold/mixed phased clouds (\cite{wallace2006atmospheric}), where the interaction of ice phase hydrometeors is a fundamental ingredient for lightning production. It is generally accepted that the electrical structure of thunderstorms is developed by updrafts that separate hydrometeors (ice \& graupel esp.) with opposite charge signs into upper and lower levels (\cite{saunders2008charge}). The transfer of electric charge results from collisions of hydrometeors, and is highly sensitive to thermodynamic conditions within which they are grown. In most severe thunderstorms associated with intense updrafts and thus large vertical extent, a sufficient frozen hydrometeor growth zone (above \SI{0}{\celsius} level) and the charge zone (0$\sim$\SI{-20}{\celsius}) are almost guaranteed. The large vertical span also provides colorful growth environments which culture hydrometeors of various properties. This variety is the fundamental reason for differed charge density on particle surface among hydrometeor types (e.g., pristine ice crystals, snow aggregates, graupel), and subsequently a transfer of charge upon particle collisions in an attempt to neutralize the charge imbalance. 

%---------------------------------------------------------------------------------------%
\section{Charging Mechanisms}
%---------------------------------------------------------------------------------------%
A question that scientists have attempted to answer since late 1800's: Exactly how is charge generated in the cloud? Countless mechanisms have been proposed. Lenard Effect (also called spray electrification) was first studied by \cite{lenard1892berdie} trying to explain charging of thundercloud by drop break-up process. But this process, as stated in \cite{saunders2008charge}, is too rare and unorganized to produce strong cloud charge structure. Ions, as introduced to the  atmosphere through cosmic rays and  ground radioactivity, has been believed to involve in cloud charging. \cite{wilson1929some} described a mechanism where falling particles acquire charge by capturing ions positioned in lower half of atmosphere due to fair weather electric field, but this mechanism was later determined to be insufficient to create electric breakdown. As the study of mesoscale storm dynamics progresses, \cite{grenet1947essai} and \cite{vonnegut1953possible} first connected the distribution of charge to storms' air motion. In their theory, positive ions on earth surface are brought to cloud top by updraft, while falling particles capture negative ions that are attracted to the cloud edge, establishing separated regions of opposite charge in cloud. However, the charging contributed by convective processes were examined to be weak and disorganized, in 3-D simulations of storm electrification in \cite{helsdon2002examination}. Later on, some mechanisms involving hydrometeor interaction (e.g., charging through freezing potential upon droplet-ice collision \cite{workman1950electrical}, charge transfer through contact potential \cite{caranti1980surface}, ice lattice dislocation that carries local positive charge \cite{keith1990further}, temperature difference between colliding ice particles \cite{latham1961generation}, charged ice splinters ejected upon freezing of supercooled liquid during Hallet-Mossop multiplication process \cite{hallett1979charge}), though proven to be insufficient to initiate lightning by self, all contributed to the development of more untenable charging mechanisms.

So far, two mechanisms are considered to have withstand the test of time: Inductive charging and Noninductive charging. Both two mechanisms are, in short, the charge separation that occurs upon the collision and rebound of two ice particles. 

Inductive charging requires a previously established electric field (initially a weak downward fair weather field), which vertically polarizes the ice particles within. When a larger particle (e.g., graupel) falls with larger fall speed, and collide into a smaller particle with smaller fall speed, the contact point is more likely to be at the lower half of the larger particle. Upon collision, the positive charge induced to the lower half of the larger particle transfer to the smaller particle. Then if rebounding, the smaller particle will carry net positive charge and be carried upward by the updraft, while the graupel precipitate to lower parts of the cloud. This charge separation then reinforces the ambient electric field, promoting more inductive charging. \cite{brooks1994experimental} carried out experiment in cloud chamber and showed significant charge transferred to an ice surface when falling through supercooled droplets. But the limitation of inductive charging mechanism remains since an observational study \cite{christian1980airborne} found that graupels could not be charged as much as what was observed, even with a maximum electric field measured in cloud. Inductive charging is considered the secondary charging mechanism that is only significant at later stages of the storm.

The non-inductive charging mechanism has been proposed and studied since late 1950s, and was, over time, accepted as the primary and initial mechanism for cloud charging. As can be inferred from its name, this mechanism does not require a background electric field that polarize ice particle vertically. Instead, scientists suggested an ice particle internal charge gradient that is inward-out, before collision occurs. The conceiving of this mechanism was inspired by some lab results \cite{takahashi1978riming} that the sign and magnitude of charge gained by graupel is dependent on ambient temperature and liquid water content. However, the lab results (e.g., \cite{takahashi1978riming}, \cite{saunders1998laboratory}, \cite{pereyra2000laboratory}, \cite{saunders2006laboratory}) of sign of charge transferred to graupel varied significantly, as shown in Fig \ref{fig:charge_reverse_curves} from \cite{saunders2008charge}. Proposed by \cite{baker1987influence}, the relative vapor diffusional growth (VDG) rate theory explained such variation. Based on Baker's results from their cold room, the current flowing to the graupel is negative when colliding ice crystals grow faster than the graupel itself through vapor deposition, and is positive for the opposite. Riming is important in this mechanism given its modulation of graupel surface condition through 1) modulating VDG rate (by providing extra vapor to and heating of graupel surface); and 2) increasing the thickness of graupel's transitional layer, which affects the mass transfer upon collision. The variation of sign reversal curses derived from different cloud chambers can be accounted for by the different setup for the ice crystals growth environment right before their contact with rimers. A more turbulent cloud environment increases the likelihood of ice crystal experiencing a spike of growth rate immediately before collision, and thus negative charge being transferred to the graupel.
%---------------------------------------------------------------------------------------%
\section{Role of Ice Crystal Growth}
%---------------------------------------------------------------------------------------%

Here is the content for section two.

%---------------------------------------------------------------------------------------%
\section{Role of Riming}
%---------------------------------------------------------------------------------------%