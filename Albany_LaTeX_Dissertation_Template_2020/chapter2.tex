%%%%%%%%%%%%%%%%%%%%%%%%%%%%%%%%%%%%%%%%%%%%%%%%%%%%%%%%%%%%%%%%%%% 
%                                                                 %
%                            CHAPTER                              %
%                                                                 %
%%%%%%%%%%%%%%%%%%%%%%%%%%%%%%%%%%%%%%%%%%%%%%%%%%%%%%%%%%%%%%%%%%% 
 
\chapter{Background}
\resetfootnote %this command starts footnote numbering with 1 again.

The vast majority of lightning occurs in cold/mixed phased clouds (\cite{wallace2006atmospheric}), where the interaction of ice phase hydrometeors is a fundamental ingredient for lightning production. It is generally accepted that the electrical structure of thunderstorms is developed by updrafts that separate hydrometeors (ice \& graupel esp.) with opposite charge signs into upper and lower levels (\cite{saunders2008charge}). The transfer of electric charge results from collisions of hydrometeors and is highly sensitive to thermodynamic conditions within which they are grown. In most severe thunderstorms associated with intense updrafts and thus large vertical extent, coincidence of the hydrometeor growth zone (colder than \SI{0}{\celsius}) and the charge zone (0$\sim$\SI{-20}{\celsius}) is almost guaranteed. The large vertical span also provides unique growth environments that culture hydrometeors of various properties. This variety is the fundamental reason for differed charge density on particle surfaces among hydrometeor types (e.g., pristine ice crystals, snow aggregates, graupel), and subsequently a transfer of charge upon particle collision in an attempt to neutralize the charge imbalance. 

%---------------------------------------------------------------------------------------%
\section{Charging Mechanisms}
%---------------------------------------------------------------------------------------%
A question that scientists have attempted to answer since late 1800's: Exactly how is charge generated in the cloud? Countless mechanisms have been proposed. The Lenard Effect (also called spray electrification) was first studied by \cite{lenard1892berdie} trying to explain charging in a thundercloud by the drop break-up process. But this process, as stated in \cite{saunders2008charge}, is too rare and unorganized to produce strong cloud charge structure. Ions, as introduced to the  atmosphere through cosmic rays and  ground radioactivity, has been hypothesized to cause cloud charging. \cite{wilson1929some} described a mechanism where falling particles acquire charge by capturing ions positioned in the lower half of atmosphere due to fair weather electric field, but this mechanism was later determined to be insufficient to result in electric breakdown. As the study of mesoscale storm dynamics progresses, \cite{grenet1947essai} and \cite{vonnegut1953possible} first connected the distribution of charge to storm air motion. In their theory, positive ions on earth surface are brought to the cloud top by updraft, while falling particles capture negative ions that are attracted to the cloud edge, establishing separated regions of opposite charge in cloud. However, the charging contributed by convective processes were examined to be weak and disorganized in 3-D simulations of storm electrification in \cite{helsdon2002examination}. Later on, other  mechanisms involving hydrometeor interaction were proposed, including charging through freezing potential upon droplet-ice collision (\cite{workman1950electrical}), charge transfer through contact potential (\cite{caranti1980surface}), ice lattice dislocation that carries local positive charge (\cite{keith1990further}), temperature differences between colliding ice particles (\cite{latham1961generation}), and charged ice splinters ejected upon freezing of supercooled liquid during Hallet-Mossop multiplication process (\cite{hallett1979charge}). However, these hypotheses are proven to be insufficient to initiate lightning by self, all contributing to the development of more untenable charging mechanisms.

So far, two mechanisms are considered to have withstand the test of time: Inductive charging and Noninductive charging. Both mechanisms are, in short, the charge separation that occurs upon the collision and rebounding of two cloud particles. 

Inductive charging requires a previously established electric field (initially a weak downward fair weather field), which vertically polarizes the ice particles within. When a larger particle (e.g., graupel) falls with larger fall speed and collides with a smaller particle with smaller fall speed, the contact point is more likely to be at the lower half of the larger particle. Upon collision, the positive charge induced to the lower half of the larger particle transfers to the smaller particle. Then if rebounding, the smaller particle will carry net positive charge and be carried upward by the updraft, while the graupel precipitate to lower parts of the cloud. This charge separation then reinforces the ambient electric field, promoting more inductive charging. \cite{brooks1994experimental} carried out experiments in a cloud chamber and showed significant charge transferred to an ice surface when falling through supercooled droplets. But the limitation of inductive charging mechanism remains since an observational study \cite{christian1980airborne} found that graupel could not be charged as much as what was observed, even with a maximum electric field measured in cloud. Inductive charging is considered the secondary charging mechanism that is only significant during later stages of the storm.

As reviewed by \cite{saunders2008charge}, the non-inductive charging mechanism has been proposed and studied since late 1950s, and was, over time, accepted as the primary and initial mechanism for cloud charging. As can be inferred from its name, this mechanism does not require a background electric field that polarize ice particle vertically. Instead, scientists suggested an ice particle internal charge gradient that is inward-out before collision occurs. The conception of this mechanism was inspired by lab results of \cite{takahashi1978riming}, identifying that the sign and magnitude of charge gained by graupel is dependent on ambient temperature and liquid water content. However, various lab results (e.g., \cite{takahashi1978riming}, \cite{saunders1998laboratory}, \cite{pereyra2000laboratory}, \cite{saunders2006laboratory}) of sign of charge transferred to graupel vary significantly, as shown in \ref{fig:ChargeSign} from \cite{saunders2008charge}. Proposed by \cite{baker1987influence}, the relative vapor diffusional growth (VDG) rate theory explained such variation. Based on Baker's results from their cold room, the charge of the current flowing to the graupel is negative when colliding ice crystals grow faster than the graupel itself through vapor deposition, and is positive for the opposite. Riming is important in this mechanism given its modulation of graupel surface condition through 1) modulating VDG rate (by providing extra vapor to and heating of graupel surface); and 2) increasing the thickness of the graupel transitional layer, which affects the mass transfer upon collision (\cite{baker1987influence}). The variation of sign reversal curves derived from different cloud chambers can be accounted for by the different setup for the ice crystal growth environment right before their contact with rimers: A more turbulent cloud environment increases the likelihood that an ice crystal experiences a spike of growth rate immediately before collision, and thus a negative charge being transferred to the graupel.
%---------------------------------------------------------------------------------------%
\section{Role of Ice Crystal Growth}
Many times, because much of the precipitation that is produced in heavy-precipitating convective systems forms and grows as frozen hydrometeors, it is important to understand and appropriately represent ice growth in models. 
Cloud ice grows through both vapor deposition and collection (aggregation and riming). Collection processes are accelerated in vertically stretched convective systems (\cite{pinsky2001collision}) such that precipitation-sized hydrometeors can easily establish and develop, which then may fall through the melting layer and to the surface.  The sizes and quantity of these hydrometeors is largely dependent on the rate of ice crystal growth through both vapor deposition and the collection that follows. It is thus crucial to track the individual growth of monomer ice crystals.

While liquid is relatively easy to model in predictive parameterizations due to its spherical shape, the complex shapes and fickle changing habits of ice crystals makes them much more difficult to represent. Crystals nucleate and grow initially through vapor deposition, the rate of which is a function of the saturated environment and the size and shape of the ice crystal itself (e.g., \cite{hallett1958influence,chen1994theoretical}). Except at temperatures T$\sim$-10 and -22$^\circ$C, it is well-known that ice crystals form and grow as either plate-like or columnar shapes, and at higher saturations, grow into dendrites or needles, respectively (e.g., \cite{hallett1958influence}). Hence, in addition to environmental properties, the rate of growth through vapor deposition relies on the vapor density gradient that is unevenly distributed above the surface of a non-spherical ice crystal, where a larger vapor density gradient is associated with a larger curvature (\ref{fig:icehabit}). This positive feedback between crystal shape and local growth rate makes the bulging parts of an ice crystal continue to bulge; the shape of the ice crystal evolves nonlinearly into extreme habits, accelerating ice particle mass (\cite{lamb1971growth, sulia2017simulateda, chen1994theoretical, sulia2011ice}), fall speed, and subsequent collection rate (\cite{sulia2021new}). 

Most parameterizations of ice crystal growth utilize observation-based empirical coefficients instead of theoretical growth mechanisms, giving unsatisfactory predictions of liquid and ice mass and lack connections to local growth environment (e.g., \cite{harrington1995parameterization, walko1995new, mitchell1996use, meyers1997new, woods2007improve, thompson2008explicit, morrison2008novel, morrison2010improved}). A novel technique in which ice particle shape is realistically initiated and evolved has been developed. The Adaptive Habit Model (AHM, \cite{harrington2013methoda}) combines parameterizations from classic capacitance theory and a mass redistribution hypothesis to appropriately evolve crystal aspect ratio, $\phi$ (Sulia and Harrington 2011) and thus capture the nonlinear feedback between $\phi$ and VDG rate.  Initial studies were completed evaluating the AHM from the aspects of single-particle growth and the evolution of a binned distribution of particles (\cite{sulia2011ice}), with subsequent investigation into the sensitivity to dynamics (\cite{sulia2013method}) following a bulk approach. Bulk analyses were extended to investigations using WRF-LES (\cite{sulia2014dynamical}), exploring the effect of crystal growth in Arctic stratiform cases, where feedbacks between cloud ice growth through liquid evaporation and thermodynamic (e.g., cloud top longwave radiative cooling) and dynamic (e.g., Turbulent Kinetic Energy) factors lead to cloud glaciation and dissipation, processes discussed in \cite{schubert1976experiments, harrington1995parameterization, ovchinnikov2011effects, morrison2012resilience}. The detailed ice information allowed \cite{ sulia2017simulateda,sulia2017simulatedb} to couple the AHM with the forward operator of \cite{ryzhkov2011polarimetric} to simulate polarimetric radar variables that are highly sensitive to ice shape and density. These studies reveal how the consideration of ice habit within clouds can affect cloud lifetime and precipitation and sheds light on how it can improve predictions of later-stage hydrometeor (e.g., aggregate and graupel) growth and its related phenomena (e.g., lightning) in convective systems.\par

Following initial growth via vapor deposition, collection via riming and aggregation is dominant for frozen hydrometeor development in thunderstorms (\cite{zeng2001microphysics}). Riming and aggregation are the main processes for graupel and snow growth from ice crystals, respectively. The evolving information on ice $\phi$, which affects estimations of the collection kernel, is indispensable for representing both processes. Based on the unique representation of ice in AHM, novel representations of growth by riming \cite{jensen2015modeling} and ice-ice aggregation \cite{sulia2021new} have been developed. These methods avoid autoconversion assumptions among hydrometeor categories since they have detailed and evolving hydrometeor information, and thus provide more accurate predictions of the distribution of hydrometeor mass in a system, required for the prediction of charge distribution and lightning production.
