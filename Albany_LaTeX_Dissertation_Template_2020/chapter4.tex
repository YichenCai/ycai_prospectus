%%%%%%%%%%%%%%%%%%%%%%%%%%%%%%%%%%%%%%%%%%%%%%%%%%%%%%%%%%%%%%%%%%% 
%                                                                 %
%                            CHAPTER                              %
%                                                                 %
%%%%%%%%%%%%%%%%%%%%%%%%%%%%%%%%%%%%%%%%%%%%%%%%%%%%%%%%%%%%%%%%%%% 
 
\chapter{Lightning and Microphysics Observation Data}
\resetfootnote %this command starts footnote numbering with 1 again.
Various observation platforms for cloud electrification, microphysics, kinematics, and thermodynamics have been utilized for crude validation of model simulations, and will be done so for this work. A temporal and spatial (latitude only) availability of selected observation data is plotted in \ref{fig:data_timeline}. Listed below are descriptions of the potential data sources including their limitations and resolutions.
%---------------------------------------------------------------------------------------%
\section{National Lightning Detection Network (NLDN)}
%---------------------------------------------------------------------------------------%
Vaisala’s U.S. National Lightning Detection Network (NLDN, \cite{cummins2009overview}) spanning continuously from January 1988 to present, is a long-standing and reliable cloud-to-ground (CG)  flash  detector  across  the  Continental  U.S.  The NLDN  consists  of  100+  ground-based  sensors and uses direction finding and time-of-arrival (TOA) methods to locate the very-low-frequency (0.4-400kHz) electromagnetic signals emitted during lightning strikes. The data files provide 2-dimensional location, polarity, and strength of strikes with minimum flash detection efficiency (DE) of 90\% and stroke location accuracy of 0.5km. Since version 3 (2008), thresholds differentiate intra-cloud (IC) flashes from CG. These ground strike data will be used to (1) provide information on the overall spatial and temporal evolution of thunderstorm events and (2) select specific flashes, to be used in other datasets for insight into individual flash structure, from which a charge structure can be inferred (\cite{mazur1997initial, thomas2001observations}).

Limitation: The NLDN provides only the horizontal location of lightning discharge and lacks information on the vertical component. The altitude of very-high-frequency (VHF, 30-300MHz) sources is helpful in identifying positive and negative charge regions, and will be acquired from other observation datasets, for example, the Lightning Mapping Array (LMA) that will be discussed in \ref{sec:LMA}. 

%---------------------------------------------------------------------------------------%
\section{Tropical Rainfall Measuring Mission (TRMM)}
%---------------------------------------------------------------------------------------%
The TRMM satellite, which served from 1997 to 2015 at ab altitude of 402.5 km and an inclination of 35$^{\circ}$, was designed to improve the understanding of the distribution and variability of precipitation within the tropics. TRMM's concurrent information on lightning from the Lightning Imaging Sensor (LIS, \cite{Blakeslee1998Lightning}) and hydrometeors from TRMM Microwave Imager (TMI, \cite{wentz2015remote}) is ideal for this study. 
\subsection{LIS}
The LIS is a lightning imager using an optics lens with a narrow-band filter (777nm). With a sampling interval of 2ms, it provides lightning data of 3 levels: ``flash," ``group," and ``event." Each ``flash" is documented with a set of ``groups;" each ``group" is composed of simultaneously and adjacently brightened pixels (i.e., ``event").  For selected cases, a 2D (lon, lat) array of \textit{Flash Extent Density (FED)} with the same horizontal resolution as model simulations will be generated using LIS scattered data of events, groups, and flashes. This array can be built by first marking the grid columns that are ``touched" by a flash (interpreted as having lightning event(s) occurred in the column), repeating the process for all flashes initiated during satellite screening time, and finally counting the marking times for each grid column. With the same shape, an array of \textit{Flash Initiation Density (FID)} will be generated by counting the times the first event(s) of flashes occurred in each grid column. Another array of \textit{Event Radiance (ER)} with the same shape will be created using the lightning\_event\_radiance field in LIS data. 

Limitation: Due to background noise, the DE of LIS varies depending on time of day and the intensity of the lightning. Thus, in an approach similar to \cite{solimine2021relationships} (in review), the lightning density mentioned above will be scaled by empirically derived DEs per local solar hour at the time of observation as provided by \cite{cecil2014gridded}.

\subsection{TRMM Microwave Imager (TMI)} 
TMI detects atmosphere-emitted radiation and differentiates cloud ice from liquid using brightness temperature anomalies. The TMI profiling algorithm (2A12) determines vertical profiles of hydrometeors within 14 layers up to 18km. Information of TMI's ice, snow, rain, and graupel water contents ($g/m^3$) and latent heating ($K/h$) will be retrieved together with LIS data. TMI horizontal resolution (4.4*5.1km) will be adjusted to match model simulations.

Limitation:  The orbital nature of TRMM results in not only a limited observing time (80s) over a location, but also potential for misalignment between the system evolution and orbital scan. This can be solved by case selection based on TRMM data availability and set model output interval to $\sim$1.5 minutes. Supplementary datasets will be used for evolutive information.
%---------------------------------------------------------------------------------------%
\section{Local Mapping Arrays (LMA)}
\label{sec:LMA}
The New Mexico Tech LMA is a unique 3-D total lightning location system (\cite{rison1999gps}). LMA adapts the TOA technique to locate VHF radiation emitted during lightning discharge, with a DE of $\sim$100\% within 0-50km. This work will apply and evaluate the LMA deployed in North Alabama (NALMA), Oklahoma (OKLMA), and Washington DC (DCLMA). All LMAs have scattered record on VHF 3-D location, time detected, power (dBW), and $\chi^2$. NALMA VHF data is reprocessed with a flash clustering algorithm and thus will be the focus of study. 4D (time, lon, lat, alt) \textit{FED, FID}, and \textit{VHF Source Density} will be generated for model comparison.

Limitation: There is a rapid drop of DE with distance from the LMA, as denoted in \cite{carey2005lightning}. This will be taken into account during analysis. Other lightning data (e.g., from GLM, NLDN, LIS) will be used as supplementary validation, qualitatively. 
%---------------------------------------------------------------------------------------%

\section{Geostationary Lightning Mapper (GLM)}
The GLM (\cite{goodman2013goes}) is deployed on Geostationary Operational Environmental Satellite (GOES-16) as a joint effort between NOAA and NASA. Onboard GLM is a 777.4nm high-speed  1372$\times$1300 CCD focal plane, whose value is further highlighted by the advanced Lightning Cluster-Filter Algorithm that processes lightning signals using their unique spatial, temporal, and spectral features. This allows GLM to operate at 36,000km yet capture around 75\% (higher than LIS at daytime) of total lightning with peak location error within 3.5km (referred to as ground-based Radiofrequency Lightning Mapping Array (LMA)). Similar arrays of \textit{FED, FID}, and \textit{ER} described above will be generated using GLM, with an added time dimension matching model output. The limitation of this dataset lies in the lack of vertical structure of lightning, which will be improved by using local lightning mappers, described above.

Limitation: The temporal span of available GLM data is shortened by its late initiation of operation (March 2017). If using this dataset, more recent cases will be selected.
%---------------------------------------------------------------------------------------%
\section{Next generation Radar (NEXRAD)}
The construction of NEXRAD network was initiated (first radar built) in June 1992 and completed in August 1997. To date, NOAA NWS operates 160 S-band NEXRAD sites across the U.S. NEXRAD provides three base data quantity: reflectivity, mean radial velocity, and spectrum width. The reflectivity data will be helpful in illustrating the structure and evolution of the mesoscale properties of selected storms. This study implements the GridRad (\cite{Bowman2017}) dataset for 3-D gridded radar data interpolated from nation-wide NEXRAD radar sites. This dataset is comprised of radar reflectivity data with horizontal resolution of $0.02^{\circ}*0.02^{\circ}*1km$ and 1 hour interval, spanning from 1995 to 2017. Also using algorithms available in GridRad, filtering (removal of bins with low reflectivity weights) and "de-cluttering" (removal of non-meteorological echo) were applied for improved data quality, as shown in \ref{fig:gridrad_compare}.

The Dual-Polarization (DP, \cite{thompson2014dual}) data has been available since 2014. Additional to the three base data mentioned, sites with DP technology provide three extra base products: differential reflectivity, correlation coefficient, and differential phase. Together they offer information on shape, density, composition of particles and the spatial variability of these quantities. This gives rise to products like Level III Hydrometeor Classification (HC), which allows information on 3-D distribution of classified hydrometeors, among many other quantities uniquely beneficial for comparisons with the AHM. The HC data at six elevation angles from $0.5^{\circ}$ to $3.4^{\circ}$ will be interpolated horizontally and vertically for selected cases. 

Limitation: The HC data does not include the quantity information of hydrometeors, only the type of dominant hydrometeor. This limits its analysis to be qualitative. Detailed information on the quantity of each type of hydrometeor can be retrieved from TMI's vertical profiling.
%---------------------------------------------------------------------------------------%