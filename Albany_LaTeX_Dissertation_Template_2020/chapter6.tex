%%%%%%%%%%%%%%%%%%%%%%%%%%%%%%%%%%%%%%%%%%%%%%%%%%%%%%%%%%%%%%%%%%% 
%                                                                 %
%                            CHAPTER                              %
%                                                                 %
%%%%%%%%%%%%%%%%%%%%%%%%%%%%%%%%%%%%%%%%%%%%%%%%%%%%%%%%%%%%%%%%%%% 
 
\chapter{Preliminary Results}
\label{chap:results}
\resetfootnote %this command starts footnote numbering with 1 again.
In this chapter, simulation results of the thunderstorm on 21 Jun 2016 introduced in \ref{chap:case} from all members in \ref{table:config} will be compared from the aspect of radar reflectivity and lightning production. Then, the two control simulations (members (1) and (7) in \ref{table:config}) will be further analyzed in detail from the aspect of synoptics, microphysics, and electrification.
%---------------------------------------------------------------------------------------%
\section{Synoptics}
%---------------------------------------------------------------------------------------%
Simulated results from the outer domain of members (1, 4km) and (7, 12km) (\ref{table:config}) compared against the NAM 12-km analysis data for synoptic performance. The mean sea level pressure remains well simulated for both 16km and 4km simulations at June 21st 12Z and 18Z after 36 and 42 hours into the simulation, respectively (\ref{fig:mslp}). At 12Z, the trough extending from Maine crossing New Jersey and down to Virginia is clearly present in both simulations. The 16km simulations (\ref{fig:mslp}, middle column) show slightly better proximity to the NAM, where the 4km runs (\ref{fig:mslp}, last column) generated a larger low pressure center over the Lake Erie region and stronger pressure ridge extending from southwest Virginia. Overall there is a slightly stronger pressure gradient in the 4km run. The deeper trough of 4km is even more obvious by 18Z. The 1010hPa isobar extends only to the southern end of NJ in the 16km run, whereas in both NAM and 4km run, the north of both Delaware and Maryland were ``touched by" or covered in 1010hPa isobar. Overall, the 4km run simulated the genesis of the secondary low pressure center over the region slightly north of DC in closer proximity to the NAM than the 16km run. This low-pressure center in 4km also appears to be more isolated than in both NAM and 16km run. 

The 2-m temperature simulated by both runs are nearly identical to the NAM analysis (\ref{fig:sfc_analysis}. A slightly stronger cold pool at the eastern West Virginia - west Virginia border was simulated by the 4km run compared to both the NAM and 16km run. The surface wind from both simulations are similar but both appear to be more convergent over central Pennsylvania as compared to the NAM, especially at 12Z. The north-northwesterly wind over NYS and PA in both runs (\ref{fig:sfc_analysis}, b-c) advected more cold air into the southeast PA, which leads to cooler temperatures over this region at 18Z relative to the NAM. The wind direction over the ocean east of the coastline is more southerly in the NAM compared to the simulations, implying a weaker warm air advection in both runs.
%---------------------------------------------------------------------------------------%
\section{Mesoscale}
%---------------------------------------------------------------------------------------%
Simulated radar reflectivity from all 12 members in \ref{table:config} are compared against NEXRAD observations. At 12Z 21 Jun (\ref{fig:radard02_21T12}), no members were able to capture in entirety the location and/or intensity of cloud and precipitation. Specifically, all 1km runs (\ref{table:config} (1)-(6)) oversimulated the convective signature in northern VA as well as in eastern Maryland and Delaware and was generally unable to resolve the signature in southeastern PA. In contrast, the 4km runs (\ref{table:config} (7)-(12)) show improvements in resolving the signatures in northern VA, southeastern PA (in some cases), and do not oversimulate the reflectivities in MD or DE to the same extent as in the 1-km case. However, the location and intensity of reflectivity simulation in the 4km case are similarly questionable. Further investigation of the inconsistencies in mesoscale simuation are analyzed using the larger domain of members (1) and (7) at this timestamp. The cluster in 1km runs is at the southern tail of the large scale banded convection, which experienced postponed eastward propagation  during the simulation. This cluster is also overly intensified in the outer domain of member (1) (rightmost of \ref{fig:radard01_21T12}), possibly due to feedback from its inner domain in the 2-way nested run. However, the 16km simulation (middle of \ref{fig:radard01_21T12}) barely resolved this banded convection at the time. It is also worth noticing that except for horizontal resolution, simulated reflectivity is the most sensitive to the use of IC/BC. Simulations using GFS analysis data (\ref{fig:radard02_21T12} (2),(8)) showed very different location of convection compared to rest of the members in their groups.

The focus of simulated radar reflectivity should be at later time of the day for the storm of interest. By 18Z (\ref{fig:radard02_21T18}), it is clear that all 1km runs captured the thunderstorm in the DC region. Simulated location of the storm is about 45km north of observation, and the simulated regional maximum dBZ is close to that observed (both simulation and observation at around 63dBZ at 19Z). The 4km simulations, on the other hand, failed to resolve the storm of interest. Simulations using the graupel version of M2M (\ref{fig:radard02_21T18}, member (3)) versus hail (member (1)) results in differing morphology of the storm, where the storm in (3) evolved in a more linear manner. When compared to graupel (3), hail (1) reaches maximum intensity and dissipates faster than (3). Graupel simulations (3) also show a larger precipitation area south of DC, which can relate to the slower sedimentation rate when graupel is considered in lieu of hail. The storm simulated by N2M (\ref{fig:radard02_21T18} (4)) acquired larger maximum intensity (about 3 dBZ larger than (1)), but with a smaller convective area. Simulations with cumulus scheme turned on (\ref{fig:radard02_21T18} (5)) results in more scattered and a less organized convection pattern, with smaller area and weaker intensity, as compared to the control simulation (1). Starting the simulation one day earlier (\ref{fig:radard02_21T18} (6)) did not deter the capture of convection initialization of the storm. The storm in (6) developed slightly earlier than in members (1)-(5), which is closer to observation given that in all 1km runs, the storm matured later than observed. 

Overall, the 4km runs represent the larger-scale reflectivity in better proximity to observation during earlier 21 Jun (before local noon), but the 1km resolution was necessary to capture the small scale afternoon thunderstorm in decent timing and intensity.

\section{Electrification}
\label{sec:electrification}
Physical simulation of lightning is performed following the implementation of the WRF-ELEC package into the M2M parameterization, as guided by the N2M scheme (\ref{chap:models}). \ref{fig:lightdis17_19} shows the comparison of the 12 simulation members in \ref{table:config} to observation from DC-LMA and NLDN during the period of the most intense lightning activity observed (17-19Z 21 Jun).  The N2M (\ref{fig:lightdis17_19} (4)) model severely overestimates the total amount of discharge in the region by ~30 times as compared to the DC-LMA. The horizontal distribution of (4), which is highly correlated to radar reflectivity in \ref{fig:radard02_21T18} (4), is more localized and linear compared to observation. The secondary maximum of lightning activity west of DC is well capture by (4), but appears to be an overestimation. The 1km M2M members (\ref{fig:lightdis17_19} (1),(2),(5),(6)), though with simulated maxima and sum much closer to the observation, exhibit much more spread out areas with lightning discharge density of ~100/$km^2$, which are absent in observation, similarly aligning with radar signatures (\ref{fig:radard02_21T18}). None of the 4km simulations produced significant amount of lightning, as can be expected from \ref{fig:radard02_21T18} (7)-(12). Members using M2M graupel version (3),(9) did not produce any lightning throughout the simulation, the details of which will be discussed later. When comparing data from DC-LMA and NLDN, it is worth noting that DC-LMA detected about 5 times more discharge in total over the region of consideration, and an local maximum about 7 times as much as that from NLDN, although NLDN capture more lightning activity outside the reach of DC-LMA (e.g., west boundary of the domain). This can be due to the differing sensitivity of detecting instruments, as well as the algorithms used in the two networks to finalize the discharge data. These differences present a challenge when comparing to model output. In the WRF-ELEC discharge scheme, the discharge source density is defined as the number of ``points" exceeding 0.01 $nC/m^3$ in the column. These ``points" can be analogous to the radiation sources detected by antennae, but there is uncertainty comparing the magnitude of the two.  Given that the source density is the most basic-level observation for lightning, further knowledge on types of other lightning observation products and physical understanding of model output are needed for more in-depth validation.

Discharge is built upon charge. The time-height (horizontal-averaged) distribution of charge carried by individual hydrometeor types are shown in \ref{fig:mem1_charge} for member (1) and \ref{fig:mem7_charge} for member (7). Note that the abrupt start of ion charge ((k,l) in both figures) is due to the fact that ion charge is only created as a result of lightning breakdown and charge released from hydrometeors, before which there is currently no explicit scheme parameterized for ion charging. Thus, the presence of ion charge can be an indicator of the first simulated discharge. In this case, the first lightning event occurred 3 hours after the start of 1km simulation and 11.5 hours for the 4km. Buildup of charge to the breakdown threshold is much more rapid in the 1km run.

For the purpose of a more comprehensive analysis of charge generated by all members, \ref{fig:charge_z1d} depicts the vertical distribution of charge sign and magnitude from all 12 members. Given the overall homogeneity of vertical distribution of charge/charging rate throughout the simulation time (\ref{fig:mem1_charge},\ref{fig:mem7_charge}), a time range with major lightning activity, 10 to 30 hours (1000 UTC 21 to 0600 UTC 22 Jun), was selected to calculate the horizontal-temporal average, with a vertical interpolation resolution of 0.5km. A subset horizontal region of the storm cell(s) of interest is considered but have not yet been conducted due to dynamic nature of the storm. From \ref{fig:charge_z1d}, fairly clear maximum and minimum are taken by mem4/10 (N2M 1km/4km, green) and mem3/9 (M2M graupel version, orange), respectively, which corresponds to the max lightning produced by mem4 and no lightning produced by mem3/9  (\ref{fig:lightdis17_19}). It is interesting to see that mem10 (N2M 4km, green dashed) exhibits even larger charge/charging rates compared to its 1km analog (\ref{fig:charge_z1d}, green solid), but with incomparably small amount of lightning produced during this 2-hour period. However, \ref{table:lgt_stat} shows that mem10 produced about 2 orders as much of discharge as that from other 4km members. Most other simulations show smaller magnitude of charge (rates), except for the charge carried by raindrops (\ref{fig:charge_z1d}b), which is much less and with opposite an sign for the N2M simulations. But also notice the overall smaller (~1 order) magnitude of rain (b) and cloud (a) charges compared to other hydrometeors (c-e): Most charges are carried by frozen hydrometeors (ice, snow, and graupel), with the maxima being negative charge of ice at 6km for N2M (12km for M2M) and positive graupel at the same level for N2M (8km for M2M). It appears that the main charging zone in N2M is near 5-7km (\ref{fig:charge_z1d} (h)). The noninductive charging shown is the summation of graupel collisional charging with ice and snow, and thus corresponds well with positive graupel charge (\ref{fig:charge_z1d} (e)) and the negative ice charge (\ref{fig:charge_z1d} (c)), with differences due to hydrometeor mass transfer. In this case, extra charge must have been transferred to graupel above the 10km level to make up an secondary positive charge zone for graupel, probably due to non-charge interaction with homogeneously freezing cloud ice. There is disparity of charge signs of ice and snow from M2M to N2M, regardless of the similar noninductive charging signs distribution (positive for 4-7km and negative for 7-10km). The opposite maxima of ice and graupel charge at 6km in N2M is nowhere to be found in any level in M2M. All M2M members simulated a negative peak for ice (15-25pC/kg) at 12km (\ref{fig:charge_z1d} (c), slightly higher for 4km runs), and another negative maximum for graupel (5-15pC/kg) at 8.5km (\ref{fig:charge_z1d} (e)). The charge carried by snow shows more divergence among M2M members, both for sign and magnitude of charge. The vertical distribution of snow charge show no obvious relation to the noninductive charging rates (\ref{fig:charge_z1d} (h)). Further analysis of microphysics processes rate and decomposed charging rate for SI,GI,GS (\ref{table:m2m_charge_type}) is required to explain such behavior, and will be a portion of future work.

\section{Microphysics}
\label{sec:micro_anal}
Charging processes are driven by the microphysics. In this section, an overview of the microphysics will be introduced, followed by attempts to explain the extremely small charging rates of the M2M graupel version.

Following \ref{fig:charge_z1d}, the horizontal-temporal mean of hydrometeor mixing ratios is computed (\ref{fig:hdmt_z1d}). Overall, the mass of frozen hydrometeors (ice, snow, and graupel/hail) dominate the liquid (cloud and rain) with a rough ratio of 4:1. Most frozen hydrometeors melt almost instaneously after reaching the melting layer and below at which the liquid hydrometeor masses reach their maximum. The two most eye-catching signatures are 1) the larger ice mass mixing ratio from the N2M 1km run ((\ref{fig:hdmt_z1d}) (c), solid green), and 2) the larger mass mixing ratio of graupel from both the 1km and 4km runs of M2M graupel version (\ref{fig:hdmt_z1d} (e), orange) relative to hail (all other colors) in the other members. N2M (\ref{fig:hdmt_z1d} (d), green) also has the largest amount of snow peaking at $\sim$1km higher than the M2M simulations. Further reasoning for these distribution structures requires an explicit and integrated analysis of process rates related to the hydrometeor types. It is interesting to see that by simply switching off the hail option in M2M, the mass of graupel more than tripled, under the case of both 1km and 4km runs (\ref{fig:hdmt_z1d} (e)). This amount of graupel generation seems rather abrupt when compared to all other members of simulation, but observation is needed for validation. Possible solution would be TRMM 2A12 with vertical profile products of hydrometeor, which is only available before April 2015. 

\subsection{Sensitivity to graupel/hail option}
The hail option in M2M increases the density of the hydrometeor of choice from 400 (graupel) to 900$kg/m^3$ (hail), and similarly increases the fall speed parameters (refer to \ref{table:m2m_param}). These are the only two adjustments made to the M2M scheme; no other change is inflicted by this option. It is worth mentioning the following tests were attempted: 1) manually increasing the graupel density to 900$kg/m^3$ without changing fall speed parameters, which resulted in no/little change, i.e., lightning discharge remained absent; 2) manually multiplying the graupel fall speed by a factor of eight throughout the simulation, which led to a lightning amount comparable to that in N2M. It is thus proposed that the fall speed of the graupel/hail hydrometeor itself is one of the dominant factors in determining the charging rate of the entire thunderstorm system. 

Analysis of simulated graupel/hail characteristics is guided by component terms of the charging equation. Combining \ref{eq:3.14}-\ref{eq:3.16} shows that the charging rate is proportional to the collision rate $n_{xy}$ and charge per collision $\delta q_{xy}$. For both terms, the fall speed of collector $V_y$ is important, indicating that the dependence of graupel fall speed to the charging rate is nonlinear. \ref{fig:GIcolli} shows a comprehensive analysis of the terms mentioned above, simulated by the original graupel-only M2M (first colummn), hail-version M2M (second column), and N2M (third column), all with 1km resolution for the inner domain. The variables are shown in the order of manipulation level, where the sedimentation rates are calculated using fall speed, and the collision rates are based upon graupel and ice number concentration and fall speed. N2M generated ice numbers about 4 times as much as M2M members, maximizing at $\sim$11km. The number of graupel, however, turns out to be largest in M2M graupel version, with a mean of ~280 $\#/kg$, followed by 85 and 15 $\#/kg$ from M2M-hail and N2M respectively, all excluding the horizontal averages less than $10^{-3} \#/kg$. The vertical maxima of M2M graupel number occur around 8-9km (\ref{fig:GIcolli} a-b), while N2M shows a weaker and lower peak at 6km (\ref{fig:GIcolli} c) that extends upward as storm intensifies. Surprisingly, the fall speed of hail (\ref{fig:GIcolli} g) and graupel (h) in M2M-hail and M2M-graupel, respectively, do not differ much with regard to mean and max values and the distribution at upper levels (above 1km), but a larger number of hail (with a larger fall speed) scavenged their way down to the surface, while M2M-graupel shows a clear region with almost no graupel near the surface (below 300m), implying that many graupel particles (potentially smaller and lighter, given its larger number mixing ratio compared to M2M-hail) remain suspended aloft. N2M deviates with even more extreme graupel fall speed with a vertical maximum maintained more persistently on their way to the surface. The mean size of graupel from all three models is required for further explanation. The sedimentation rate, which represents the rate of graupel removal due to sedimentation, is larger for M2M-graupel than M2M-hail, which can be explained by the larger number concentration of graupel than hail (\ref{fig:GIcolli} a-b) at the lower edge of their vertical maxima, leading to more particles for sedimentation in M2M-graup. The collision rate of graupel with cloud ice is (\ref{fig:GIcolli} m-o) from all three members is quite different. N2M shows a very different and interesting multi-modal vertical distribution. This is primarily due to the difference in the vertical maxima of number of graupel (6km) and ice (10.8km), as well as the much larger number of ice, which results in multiple maxima for graupel-ice interactions. This signature is weakly present in M2M simulations (\ref{fig:GIcolli} m-n; notice the secondary "revitalization" of the collision rate at 5-6.5km for M2M, but not as significant due to the more spread-out concentration of graupel and smaller amount of ice in M2M (\ref{fig:GIcolli}, a-b,d-e). 